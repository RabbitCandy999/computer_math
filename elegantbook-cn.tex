\documentclass[lang=cn,newtx,10pt,scheme=chinese]{elegantbook}

\title{数学探秘}
\subtitle{关于计算机数学的思考}

\author{糖糖 \& 小慧}
\date{2024/2/18}
\version{4.5}

\setcounter{tocdepth}{3}

% \logo{logo-blue.png}
\cover{cover.jpg}

% 本文档命令
\usepackage{array}
\newcommand{\ccr}[1]{\makecell{{\color{#1}\rule{1cm}{1cm}}}}

% 修改标题页的橙色带
\definecolor{customcolor}{RGB}{32,178,170}
\colorlet{coverlinecolor}{customcolor}

\begin{document}

\maketitle

\frontmatter
\tableofcontents

\mainmatter
	
\chapter{初等数论与多项式}
数论的本质是研究整数之间的关系,我们可以知道,加法、减法、乘法对于整数是一个封闭的运算. 但是两个整数之间的除法不一定是一个整数,所以,数论中的许多问题都是在研究两个数之间的除法.

然而多项式的定理也与数论的定理相似,所以在这里,我们会把多项式的定理与数论的定理进行比较.

\section{整除的概念及其基本性质}
\begin{definition}[整数整除的定义]
对任给的两个整数$a,b$,其中$a\neq 0$,如果存在整数$q$,使得$b=aq$,那么称$b$能够被$a$整除(或者$a$整除$b$),记作$a\mid b$,否则,称$a$不整除$b$,记作$a\nmid b$.
\end{definition}

\begin{definition}[整数的因数与倍数]
如果$a\mid b$,那么称$a$为$b$的因数,$b$为$a$的倍数.
\end{definition}

利用整除的定义,可以非常容易地推导出下面经常能用到的一些性质.

\begin{property}
如果$a\mid b$,则$a\mid (-b)$,反过来也成立;如果$a\mid b$,则$(-a)\mid b$,同样反过来也成立.
\end{property}

\end{document}
